

\documentclass{resume} % Use the custom resume.cls style
\usepackage{hyperref}


 \hypersetup{
     colorlinks=true,
     linkcolor=red,
     filecolor=red,
     citecolor = black,      
     urlcolor=magenta,
     } 


\usepackage{multicol}
\usepackage{url}
\usepackage{bibentry}
\usepackage{natbib}
\usepackage[T1]{fontenc}
\usepackage[left=0.75in,top=0.6in,right=0.75in,bottom=0.6in]{geometry} % Document margins
\usepackage[symbol]{footmisc}


\renewcommand{\thefootnote}{\fnsymbol{footnote}}

  
\pagestyle{plain} 


\begin{document}

  
\nobibliography{ref}
\bibliographystyle{abbrv}

  
  
\begin{center}
    {\LARGE \bf Sashwat Tanay} \\  
\end{center}



204, Lewis Hall                                                      \hfill {stanay@go.olemiss.edu} \\
University of Mississippi                                       \hfill  \href{https://sashwattanay.github.io/sashwat_site}{sashwattanay.github.io/sashwat\_site} \\
University, MS 38677-1848, USA                          \hfill     ORCID: \href{https://orcid.org/0000-0002-2964-7102}{0000-0002-2964-7102}       \\

%----------------------------------------------------------------------------------------
%	EDUCATION SECTION
%----------------------------------------------------------------------------------------
\begin{rSection}{Education}
{\bf Ph.D. (Physics) } University of Mississippi \hfill {2016-2022}  \\
\hspace*{1cm} {\bf Advisor:} Prof. Leo C. Stein \\
\hspace*{1cm} {\bf Dissertation title:} Post-Newtonian dynamics of eccentric, spinning binary\\
\hspace*{4.6 cm} black holes and the associated gravitational waveforms \\
\hspace*{1cm} {\bf Courses:} One course each on post-Newtonian and black-hole perturbations; \\
\hspace*{2.8cm} two courses on quantum fields

{\bf B.Tech. (Mechanical Engineering) } Indian Institute of Technology Ropar \hfill {2009-2013}  

\end{rSection}



  
%----------------------------------------------------------------------------------------
%	WORK EXPERIENCE SECTION
%----------------------------------------------------------------------------------------

\begin{rSection}{Teaching \& Work Experience}

{\bf Teaching Assistant } University of Mississippi \hfill 2016-present  \\
\hspace*{1cm} Lab teaching and grading for Phys 221 and 222 (introductory physics) 


{\bf Junior Research Fellow} Tata Institute of Fundamental Research, Mumbai \hfill 2013-2015 


\end{rSection}


%----------------------------------------------------------------------------------------
%	HONORS & AWARDS SECTION
%----------------------------------------------------------------------------------------

\begin{rSection}{Honors \& Awards}

{\bf Graduate School Honors Fellowship}, Univ. of Mississippi (\$12,000 in total)               \hfill 2016-2020 


{\bf Junior Research Fellowship}, Tata Institute of Fundamental Research, Mumbai \hfill 2013-2015 


\end{rSection}
  
%----------------------------------------------------------------------------------------
%	PUBLICATIONS SECTION
%----------------------------------------------------------------------------------------



\begin{rSection}{Publications}
  \begin{enumerate}
  	\item \bibentry{tanay2021actionangle}
  	\item \bibentry{Cho:2021oai}
    \item \bibentry{Tanay:2020}
    \item \bibentry{Tanay:2019}
    \item \bibentry{Tanay:2016}
  \end{enumerate}
  \end{rSection}
 

\newpage
 
%----------------------------------------------------------------------------------------
%	RESEARCH INTERESTS AND EXPERIENCE
%----------------------------------------------------------------------------------------

\begin{rSection}{Research experience \& interests}

\textbf{Past work} \textbullet~ Theoretical study of the dynamics of binary black holes (BBHs) under the post-Newtonian Hamiltonian framework within
Einstein's general relativity (GR) \textbullet ~Modeling their trajectories and gravitational waves (GWs) emitted by them \textbullet ~ Exploring
preliminary data analysis implications with these waveforms. \\
\textbf{Ongoing work} \textbullet ~Compute 2PN action-angles of BBHs with arbitrary masses, spins and eccentricity
\textbullet ~Studying the rate of change of quasi-normal mode frequencies of a Kerr BH with respect to its spin
 \textbullet ~Construct the power spectrum of scalar and tensor fluctuations sourced by many-field inflationary models of cosmology
% \textbullet ~Perform a Dirac-Hamiltonian constraint analysis for a higher-derivative theory of gravity beyond GR (the dynamical Chern-Simons gravity)
%with the use of Dirac brackets

\end{rSection} 
  
  

 


%----------------------------------------------------------------------------------------
%	INVITED TALKS
%----------------------------------------------------------------------------------------

\begin{rSection}{Invited talks}


% \begin{itemize}

Montana State University (planned)       \hfill Apr 2022

Albert Einstein Institute Potsdam, ACR Seminar \hfill Jun 2021  

Simon Fraser University, Cosmology Seminar \hfill Sep 2020


% \item Indian Institute of Technology Ropar, Physics Department Colloquium \hfill Jan 2020
% \end{itemize}


\end{rSection}


\iffalse

%----------------------------------------------------------------------------------------
%	CONTRIBUTED TALKS
%----------------------------------------------------------------------------------------

\begin{rSection}{Contributed talks}


% \begin{itemize}

APS April Meeting 2022, New York (planned)  \hfill Apr 2022

Midwest Relativity Meeting (UIUC)      \hfill Nov 2021

APS April Meeting 2021 (online) \hfill Apr 2021

APS April Meeting 2020 (online) \hfill Apr 2020



% \end{itemize}


\end{rSection}

\fi


%----------------------------------------------------------------------------------------
%	SUPERVISION
%----------------------------------------------------------------------------------------

\begin{rSection}{Mentoring}




 {\bf Subhayu Bagchi}  {\footnotesize  (current grad, Univ. of Mississippi)}                                                                         \hfill Aug 2021-present  \\
{\footnotesize 2PN closed-form solution of spinning BBHs with arbitrary parameters}    

{\bf Andrew Lackie}    {\footnotesize  (current undergrad, Georgia Tech.)}                                                                      \hfill Sep 2020-present  \\
{\footnotesize Effect of periastron advance on detection errors of BBHs using Fourier domain GW templates}



 
\end{rSection}



   
%----------------------------------------------------------------------------------------
%	COMPUTER SKILLS
%----------------------------------------------------------------------------------------

\begin{rSection}{Computer skills}



\textbullet~Mathematica, C/C++, Python, Matlab, Fortran, Jekyll, Bash 
\textbullet~Github: \href{https://github.com/sashwattanay}{github.com/sashwattanay}



\end{rSection}



%----------------------------------------------------------------------------------------
%	OUTREACH
%----------------------------------------------------------------------------------------

\begin{rSection}{Outreach \& service}

Judge at The Speaker’s Edge Competition 2022 - Univ. of MS (planned)

Organized STEM Summer Camp - Univ. of MS  (2018, 19)  

Organized Spooky Physics Night - Univ. of MS (2016-18) 


 
 



\end{rSection}



%----------------------------------------------------------------------------------------
%	LANGUAGES
%----------------------------------------------------------------------------------------

\begin{rSection}{Languages}


Hindi (native), English (fluent), German (elementary)


\end{rSection}




\iffalse
%----------------------------------------------------------------------------------------
%	References
%----------------------------------------------------------------------------------------

\begin{rSection}{References}


\begin{multicols}{3}
\begin{itemize}
\item  Leo C. Stein      \\University of Mississippi\\ \href{lcstein@olemiss.edu}{lcstein@olemiss.edu} \\ \\
\item Emanuele Berti\\Johns Hopkins University\\ \href{berti@jhu.edu}{berti@jhu.edu}     \\ \\
\item Achamveedu Gopakumar\\TIFR Mumbai\\ \href{gopu@tifr.res.in}{gopu@tifr.res.in}    \\ 
\end{itemize}
\end{multicols}


\end{rSection}

\fi





\end{document}
